\documentclass[10pt,twocolumn]{article}

% use the oxycomps style file
\usepackage{oxycomps}
\usepackage{float}

\pdfinfo{
    /Title (Ethics Paper)
    /Author (Oliver Wilkins)
}

% set the title and author information
\title{Ethics Paper}
\author{Oliver Wilkins}
\affiliation{Occidental College}
\email{wilkins@oxy.edu}

\begin{document}

\maketitle

\section{Introduction}
In this paper I will argue against the possibility of completing my comps project without any ethical concerns. My project is a simulation meant to contribute to empirical literature on the quantification of human psychological phenomena, in particular the coefficient of loss aversion.\footnote{Loss aversion is defined as the tendency of people to dislike a loss more than they like an equivalently sized gain. The loss aversion coefficient measures the degree to which this is true.} The ethical issues that emerge will be different from those of most other comps projects. There is no user-facing end product, meaning that issues of accessibility are largely irrelevant. There is no individual level or sensitive data used, negating the possibility of issues around privacy and data ownership. There is no obvious way by which my project could be directly monetized, removing the chance of a potential conflict between revenue generation and any other goal of the project. Rather, the ethical issues discussed in this paper will stem from the facts that empirical research is used by policymakers to make decisions, and, especially in the case of research using simulation-based techniques, requires making subjective decisions which inevitably skew the results. In keeping with this framework, I will first describe ways in which the type of empirical research to which my project will contribute is used in the creation of public policy, and in doing so argue that the accuracy of the research is important for maximizing social welfare. I will then describe some of the subjective decisions I anticipate having to make and how those decisions might influence the results I obtain. The combination of subjectivity and the social welfare implications of the results necessarily implies that there is no possibility of completing my project with no ethical concerns.\footnote{Of course, I do not realistically expect the findings of my project to by used by any maker of public policy, and in that sense it may very well be possible to complete my project with no ethical concerns. However, for the purposes of this paper I will assume the possibility of it being used.}

\section{Why Accuracy Matters for Social Welfare}
One way in which empirical research into loss aversion is relevant to policymakers is that it plays a role in the calculation of optimal levels of redistribution. The standard\footnote{i.e. assuming everyone is 'rational', not taking into account psychological factors} model of doing this finds the level of redistribution which maximizes some welfare function, often the utilitarian welfare function which simply adds together the utilities\footnote{Utility is an economic term meant to quantify the amount of satisfaction which a person obtains from a certain ‘state of the world’} of each person in the economy. It does so by trading off between the benefit and cost of redistribution, as defined by the welfare function. The benefit of redistribution comes from the fact that economists assume individual’s utility functions over wealth to be concave, i.e. a poor person’s utility increase from receiving \$100 is greater than a that of a rich person. The cost is that the taxation required to run the redistributive program creates a work disincentive, in effect decreasing the total wealth of the society. Crucially, this model is reference-independent, meaning utility is only measured over net wealth as opposed to considering how changes relative one’s reference point of their pre-redistribution wealth may have an additional impact on utility. Loss aversion comes into play when the model allows for reference dependence, a change which is increasingly being made as the field of behavioral economics grows. See the 2023 paper by Firpo et. al. for a recent example.\footnote{Sergio Firpo, Antonio F. Galvao, Martyna Kobus, Thomas Parker, Pedro Rosa-Dias, Loss aversion and the welfare ranking of policy interventions, Journal of Econometrics, 2023, 105643, ISSN 0304-4076, https://doi.org/10.1016/j.jeconom.2023.105643.} Under this new model, loss aversion means that those who are being taxed will experience high gain-loss utility, i.e. the portion of utility which is derived from changes away from a reference point rather than net outcomes, relative to those who are receiving transfer payments. This implies that the optimal level of redistribution under a reference-dependent model is lower than under a standard, reference-independent model. The empirical literature on loss aversion therefore impacts policy in that its estimations of the loss aversion coefficient are a factor in the calculations of the optimal level of redistribution. 

While the link between empirical estimations of the loss aversion coefficient and calculations of the theoretically optimal level of redistribution are straightforward, the link between these calculations and actual ‘welfare’ as defined in the colloquial sense of the word is far less so. Firstly, we must consider the degree to which theoretical models actually influence policy. While it is true that governments employ economists to create models such as the one described in the above paragraph, it is politicians who actually decide on policy, and the degree to which they listen to the economists varies. There are many factors which go into the decisions politicians make besides what is theoretically optimal. Beyond this procedural issue, there is the philosophical question of which utility function the model should use in the first place. While mainstream economists tend to prefer a utilitarian model, this is far from the only option. The Rawlsian utility function is one which considers only the utility of the worse off person in society, with the result being a larger degree of redistribution than under a utilitarian model. In the opposite direction, libertarians take issue with the idea of utility being cardinal as opposed to solely ordinal, with the implication being that utility cannot be compared across people and therefore no redistribution should occur. There are infinitely many possible utility functions one could create. While this complicates the relationship of optimal level of redistribution to ‘welfare’ as colloquially defined, if there exists a utility function which is ‘correct,’ utility will be maximized under that function only when the value of the loss aversion coefficient, as well as the other inputs, are accurate. Therefore, the accuracy of research into loss aversion does matter in some real sense, even if the connection is not straightforward. 

Another way in which estimates of loss aversion are relevant to policymakers is in scenarios where the policymaker feels that loss aversion is causing people to make a mistake, defined as doing something which goes against their rational self-interest. In their 2012 paper, Milkman et. al. describe a thought experiment which helps to get a sense of how this might play out.\footnote{Katherine L. Milkman, Mary Carol Mazza, Lisa L. Shu, Chia-Jung Tsay, Max H. Bazerman, Policy bundling to overcome loss aversion: A method for improving legislative outcomes, Organizational Behavior and Human Decision Processes, Volume 117, Issue 1, 2012, Pages 158-167, ISSN 0749-5978, https://doi.org/10.1016/j.obhdp.2011.07.001.} 

\textit{Imagine a local legislature faced with two unpopular pieces of legislation during an economic downturn. Suppose one unpopular bill under consideration would increase government spending by 10 million dollars at a time when deficits were soaring but would create 100 new, permanent jobs. Detractors would likely focus on the lost government dollars and feel that these losses were too great to bear. Suppose the second unpopular bill involved a budget-cutting measure that would eliminate 90 government jobs, reducing the deficit by 12 million dollars. The opposition would likely contest the lost jobs in a difficult economic climate.}

Both of these bills are unpopular because loss aversion causes people to focus more on the negative aspect of each than on the positive. As a result, neither bill would pass. However, if they were both to pass the net effect would be a gain of 10 jobs and a deficit decrease of \$2 million, a result which nearly everybody would approve of. Therefore, Milkman et. al. argue, the policymakers writing these bills would improve social welfare if they combined the bills together into one. In this way, an understanding of loss aversion and a subsequent implementation of its implications into policy decisions would be helpful in preventing people from falling prey to their own mistakes. 

While the above example is highly stylized and therefore only relies on the existence of loss aversion rather than any quantification of its magnitude, in the real world a quantification would often be necessary in order to determine an appropriate policy to stop people from making decisions which are against their own rational self-interest. Here I believe it is important to note that it is necessary to be careful about what decisions are classified as going against one's rational self-interest. Specifically, a person being loss averse should not itself be considered irrational, even if under standard economic theory it is classified as such. Rather, I believe that for loss aversion to be detrimental to the point where the government should intervene it must be that the loss aversion diminishes when either the time horizon or the framing of the decision in question is changed. 

\section{Why Inaccuracy is Inevitable}
    Given that having accurate estimations of loss aversion is important for social welfare, it is necessary to discuss the degree to which true accuracy is actually achievable, or even real. In any empirical estimation of loss aversion, the researcher must make decisions about what data to use, and it is in these decisions where subjectivity plays a role. There are generally two approaches to generating estimates or parameters such as loss aversion: experimental or quasi-experimental studies, and simulations in which parameters are calibrated such that the simulation matches real-world outcomes. 
    
    The most common way to generate estimates is by running experiments. This approach has the same accuracy concerns as any other scientific study: sample size, degree to which the participants are representative of the true population, and volunteer bias, among others. In addition to these standard concerns, much of the existing literature is uniquely flawed in a number of other ways. Firstly, the populations being studied are often non-diverse, as many experiments are simply conducted by professors on their undergraduate classes. Additionally, the experiments are often conducted with either low or hypothetical stakes, and it is plausible that their results are therefore not externally valid. The external validity concern is worsened by the possibility that loss aversion is context-dependent.

    Using a simulation technique to calibrate the loss aversion coefficient to empirical data helps to solve many of the issues of the experimental approach, but also introduces new concerns. By calibrating the parameter to aggregated data, it largely eliminates the concern over unrepresentative populations being sampled. Additionally, the stakes of the decisions making up the data are real and significant. However, a simulation-based technique creates more subjective decisions which must be made. For my comps, I will have to decide on which systems to simulate. Additionally, I will have to make assumptions in order to set up each simulation, which will almost certainly be flawed in some way. I will then have to decide on a set of empirical measurements based on which I evaluate the results of my simulations, and come up with a way to aggregate the various evaluation metrics into a single score so as to be able to decide on a single estimate. Each of these decisions introduces the potential for inaccuracy.

    Beyond the subjective decisions I will have to make, there is the more fundamental question of whether the loss aversion coefficient is a 'real' thing. Certainly it is not a real thing at the population level, given that there is variation across people both randomly and across various demographics, such as income and the social structure of one's society. Even for a given person, the loss aversion coefficient still is not truly real. People's tendency to overweight losses relative to gains is the result of many factors, including genetic predispositions, cultural conditioning, and context specific considerations. In this sense it is not 'real' in that it is an aggregation of multiple causal effects, and is not itself casual. 
\end{document}
